                %% start of file `template.tex'.
%% Copyright 2006-2013 Xavier Danaux (xdanaux@gmail.com).
%
% This work may be distributed and/or modified under the
% conditions of the LaTeX Project Public License version 1.3c,
% available at http://www.latex-project.org/lppl/.


\documentclass[11pt,a4paper,sans]{moderncv}        % possible options include font size ('10pt', '11pt' and '12pt'), paper size ('a4paper', 'letterpaper', 'a5paper', 'legalpaper', 'executivepaper' and 'landscape') and font family ('sans' and 'roman')

% modern themes
\moderncvstyle{banking}                            % style options are 'casual' (default), 'classic', 'oldstyle' and 'banking'
\moderncvcolor{blue}                                % color options 'blue' (default), 'orange', 'green', 'red', 'purple', 'grey' and 'black'
%\renewcommand{\familydefault}{\sfdefault}         % to set the default font; use '\sfdefault' for the default sans serif font, '\rmdefault' for the default roman one, or any tex font name
%\nopagenumbers{}                                  % uncomment to suppress automatic page numbering for CVs longer than one page

% character encoding
\usepackage[utf8]{inputenc}                       % if you are not using xelatex ou lualatex, replace by the encoding you are using
%\usepackage{CJKutf8}                              % if you need to use CJK to typeset your resume in Chinese, Japanese or Korean

% adjust the page margins
\usepackage[scale=0.75]{geometry}
%\setlength{\hintscolumnwidth}{3cm}                % if you want to change the width of the column with the dates
%\setlength{\makecvtitlenamewidth}{10cm}           % for the 'classic' style, if you want to force the width allocated to your name and avoid line breaks. be careful though, the length is normally calculated to avoid any overlap with your personal info; use this at your own typographical risks...

\usepackage{import}

% personal data
\name{Joseph}{Priou}
\title{Curriculum Vitae}                               % optional, remove / comment the line if not wanted
\address{8, rue Michal, Paris 75013}{}{}% optional, remove / comment the line if not wanted; the "postcode city" and and "country" arguments can be omitted or provided empty
\phone[mobile]{+33 6 88 18 64 31}                   % optional, remove / comment the line if not wanted
%\phone[fixed]{01234 123456}                    % optional, remove / comment the line if not wanted
%\phone[fax]{+3~(456)~789~012}                      % optional, remove / comment the line if not wanted
\email{joseph0priou@gmail.com}                               % optional, remove / comment the line if not wanted
%\homepage{www.myname.webs.com}                         % optional, remove / comment the line if not wanted
%\extrainfo{additional information}                 % optional, remove / comment the line if not wanted
%\photo[64pt][0.4pt]{picture}                       % optional, remove / comment the line if not wanted; '64pt' is the height the picture must be resized to, 0.4pt is the thickness of the frame around it (put it to 0pt for no frame) and 'picture' is the name of the picture file
%\quote{Some quote}                                 % optional, remove / comment the line if not wanted

% to show numerical labels in the bibliography (default is to show no labels); only useful if you make citations in your resume
%\makeatletter
%\renewcommand*{\bibliographyitemlabel}{\@biblabel{\arabic{enumiv}}}
%\makeatother
%\renewcommand*{\bibliographyitemlabel}{[\arabic{enumiv}]}% CONSIDER REPLACING THE ABOVE BY THIS

% bibliography with mutiple entries
%\usepackage{multibib}
%\newcites{book,misc}{{Books},{Others}}
%----------------------------------------------------------------------------------
%            content
%----------------------------------------------------------------------------------
\begin{document}
%\begin{CJK*}{UTF8}{gbsn}                          % to typeset your resume in Chinese using CJK
%-----       resume       ---------------------------------------------------------
\makecvtitle

\small{Passionné de nouvelles technologies, je me forme dans de nombreux domaines : Mathématiques, Informatique, Économie et Sciences Sociales... Je poursuis actuellement mes études à l'université et à 42.}

\section{Éducation}

\vspace{5pt}

\subsection{Qualifications}

\vspace{5pt}

\begin{itemize}
	
	\item{\cventry{2016--2018}{Diplôme universitaire}{Paris 7 (En cours)}{Diderot}{\textit{Double licence Mathématiques-Informatiques}}{}}
	
	\item{\cventry{2017--2018}{Architecte en technologie du numérique}{42 Paris (En cours)}{Paris}{}{}}
	
\end{itemize}

\vspace{2pt}

\subsection{Projets Notables}

\vspace{5pt}

\begin{itemize}
	
	\item{\textbf{jeexstudio.96.lt}  \textit{'Développement un site web "from scratch"'}
		
		\vspace{3pt}
		
		\small{Dans le cadre d'un projet d'apprentissage des langages du web, à savoir HTML, CSS, PHP, MySQL, JavaScript. Le but ici était de recréer une structure stable et réutilisable sans utiliser aucun framework ni aucun code source externe quelconque. \href{http://jeexstudio.96.lt/Pagepremierevisite/JeexStudio.php}{(Lien jeexstudio)}}}

	\vspace{6pt}
	
	\item{\textbf{My awesome Libft}  \textit{'Développement d'une libft polymorphique, auto testable'}
		
		\vspace{3pt}
		
		\small{'My awesome libft' est un projet personnel qui donne suite à l'un des premiers projets de 42 nommé libft qui consiste à coder des fonctions qui peuvent servir dans toute une carrière de programmeur C. Ce projet est polymorphique, auto-testable et auto-portable.  \href{https://github.com/FauconFan/my_awesome_libft}{(Lien 'My awesome libft')}}}
	
\end{itemize}

\section{Expérience professionnelle}

\vspace{6pt}

\begin{itemize}

\item{\cventry{Juin 2017--Juillet 2017}{Construction d'un système de helpers pour des requêtes SQL Android}{Nuesoft}{Atlanta}{}{\vspace{3pt}J'ai suivi un mini-stage qui consistait à la base à la résolution de bugs sur l'application Android NueMD de l'entreprise NueSoft, mais rapidement (et parce que je m'ennuyais), j'ai proposé d'améliorer leur système de génération de requêtes SQL sur Android.}}

\vspace{6pt}

\item{\cventry{--}{Sites Web}{Partenaires Particuliers}{Paris}{}{\vspace{3pt}Création de sites à la demande. Les plus notables sont :
\begin{itemize}
	\item \href{https://arbredevie.net/}{arbredevie.net}
	\item \href{http://raphaelmonne.96.lt/}{raphaelmonne.96.lt}
\end{itemize}
}}

\end{itemize}

\newpage

\section{Compétences Techniques et Personnelles}

\vspace{6pt}

\begin{itemize}

\item \textbf{Languages de programmation:} \\ Maîtrise de : C, Python, Java, TeX, HTML, CSS, PHP, SQL, JavaScript \\ Quelques notions basiques en : Matlab, Python, Haskell, Matlab.

\vspace{6pt}

\item \textbf{Compétences Industrielles} Gestion/Création de frameworks. Utilisation de Travis CI. Gestion/Administration d'une équipe.

\vspace{6pt}

\item \textbf{Autres:} Esprit d'équipe. Rigoureux. Autonome. Sportif d'occasion. Peut écrire des rapports structurés et organisés.

\end{itemize}

\section{Intérêts et activités}

\vspace{6pt}

\begin{itemize}

\item{Je suis membre d'une organisation universitaire de programmation, organisé et maintenu par Yann Régis Gianas, dans laquelle nous échangeons à propos de nos projets personnels ou de sujets informatiques de manière générale. "Pourquoi le mot clé 'null' a-t-il été inventé ?" ou encore "Vim ou Emacs ?"}

\vspace{6pt}

\item{J'ai fait énormément de sport (2 ans de foot, 2 ans de rugby, 1 an d'escrime, 8 ans de judo, 10 ans de basket), je faisais parfois jusqu'à trois sports en club dans la même année montant ainsi jusqu'à 12 heures de sport hebdomadaire.}

\end{itemize}

\section{Références professionnelles}

\vspace{6pt}
 
\begin{itemize}

\item{Peut fournir 3 références professionnelles sur demande.}

\end{itemize}

% Publications from a BibTeX file without multibib
%  for numerical labels: \renewcommand{\bibliographyitemlabel}{\@biblabel{\arabic{enumiv}}}% CONSIDER MERGING WITH PREAMBLE PART
%  to redefine the heading string ("Publications"): \renewcommand{\refname}{Articles}
\nocite{*}
\bibliographystyle{plain}
%\bibliography{publications}                        % 'publications' is the name of a BibTeX file

% Publications from a BibTeX file using the multibib package
%\section{Publications}
%\begin{itemize}
%	\item Aucune
%\end{itemize}
%\nocitebook{book1,book2}
%\bibliographystylebook{plain}
%\bibliographybook{publications}                   % 'publications' is the name of a BibTeX file
%\nocitemisc{misc1,misc2,misc3}
%\bibliographystylemisc{plain}
%\bibliographymisc{publications}                   % 'publications' is the name of a BibTeX file

%-----       letter       ---------------------------------------------------------

\end{document}


%% end of file `template.tex'.